\documentclass{article}
\usepackage[utf8]{inputenc}
\usepackage{listings}
\usepackage{xcolor}
\usepackage{geometry}
\usepackage[russian]{babel}
\geometry{
    a4paper,
    total={170mm,257mm},
    left=20mm,
    top=20mm,
}

\lstset{
    language=C++,
    basicstyle=\ttfamily,
    keywordstyle=\color{blue},
    commentstyle=\color{green!60!black},
    stringstyle=\color{red},
    showstringspaces=false,
    tabsize=4,
    breaklines=true,
    breakatwhitespace=true,
}

\begin{document}

\section*{Документация}

\subsection*{Введение}
Эта документация описывает реализацию класса multigraph и различные алгоритмы, которые могут быть выполнены на нем. Код, приведенный в \texttt{main.cpp } файле демонстрирует использование класса multigraph и следующих алгоритмов:

\begin{itemize}
    \item Depth-First Search (DFS) для нахождения компонентов и эйлеровых циклов в мультиграфе.
    \item Breadth-First Search (BFS) для поиска дополнительных путей в остаточной сети.
    \item Алгоритм сопоставления для нахождения дополняющих цепочек в сопоставлении.
\end{itemize}

\subsection*{Class \texttt{Edge}}
Этот класс представляет ребро в мультиграфе. Он обладает следующими атрибутами:

\begin{itemize}
    \item \texttt{source} -исходная вершина ребра.
    \item \texttt{destination} - конечная вершина ребра.
    \item \texttt{capacity} - пропускная способность ребра.
\end{itemize}

\subsection*{Class \texttt{MultiGraph}}
Этот класс представляет собой мультиграф. Он имеет следующие приватные атрибуты:
\begin{itemize}
    \item \texttt{numVertices} - количество вершин в мультиграфе.
    \item \texttt{adjacencyList} - представление списка смежности мультиграфа.
\end{itemize}

Класс предоставляет следующие публичные методы:

\begin{itemize}
    \item \texttt{MultiGraph(int numVertices)} - конструктор для создания мультиграфа с заданным количеством вершин.
    \item \texttt{void addEdge(int source, int destination, int capacity)} - добавляет ребро к мультиграфу.
    \item \texttt{vector<int> findMaximumClique()} - находит максимальную клику в мультиграфе.
    \item \texttt{int findMaxFlow(int source, int sink, vector<int>\& path)} - находит максимальный поток в сети, используя алгоритм Форда-Фалкерсона.
    \item \texttt{vector<int> findEulerianCycle()} - находит эйлеровский цикл в мультиграфе.
    \item \texttt{bool isEulerian()} - проверяет, имеет ли мультиграф цикл Эйлера.
    \item \texttt{vector<vector<int>> findStronglyConnectedComponents()} - находит сильно связанные компоненты в мультиграфе, используя алгоритм Косарайю.
\end{itemize}

Для получения подробных объяснений каждого метода, пожалуйста, обратитесь к комментариям к коду в \texttt{main.cpp }.

\end{document}
